%%%%%%%%%%%%%%%%%%%%%%%%%%%%%%%%%%%%%%%%%%%%%%%%%%%%%%%%%%%%%%%%%
\chapter{CONCLUSION}\label{ch:conclusion}
%%%%%%%%%%%%%%%%%%%%%%%%%%%%%%%%%%%%%%%%%%%%%%%%%%%%%%%%%%%%%%%%%

In this thesis, after a brief introduction about LPWAN technologies, LoRa modulation basics and spreading factor assignment issue is discussed. We present an open source discrete event simulator which is developed from scratch to study network performance of LoRaWAN and evaluate different spreading factor assignment schemes. Moreover, we show how same spreading factor collisions can be avoided, hence, we propose a machine learning based solutions called smart DTC scheme and smart SVM scheme. We present simulation results for the lowest spreading factor assignment scheme and the proposed schemes. Simulation results show that, the proposed smart schemes can increase network performance for LoRaWAN networks, especially, when the nodes are deployed close to gateways.

TODO add summary 2 more paragraph

As for future work, transmit power optimization can be included to the proposed smart schemes. In this thesis, it is assumed that nodes always use maximum transmit power for uplink transmission, however nodes close to gateways can decrease transmit power to save energy. This will make transmissions more vulnerable to interference thus requires extra care. Also other proposed SF assignment schemes described in Chapter \ref{ch:related_works} can be integrated to simulation tool for comparing other proposed SF assignment schemes. In this thesis, it is also assumed that only LoRaWAN Class A end nodes are present, however Class B and Class C end node behaviour can be integrated to check how extra downlink communication effects smart SF schemes. Moreover, other machine learning methods can be investigated for spreading factor assignment enhancement. Reinforcement learning could be a good candidate. Also, other transmission parameters such as node id and transmission time can be included to the proposed scheme in order to improve prediction performance.
